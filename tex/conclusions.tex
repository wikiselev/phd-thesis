\chapter{Conclusions and directions for future work}

\section{Conclusions}

The main objective of this thesis was:

\begin{quotation}
\begin{spacing}{1.0}
\emph{To extend the understanding of the electrostatic contribution to the membrane lateral dynamics of proteins and lipids by means of novel computational modeling tools.}
\end{spacing}
\end{quotation}

\vspace{-0.23in}

This objective has been fulfilled by the work described in the previous chapters. In particular:

\begin{enumerate}
 \item The novel MCA for studying electrostatic effects in the membrane lateral dynamics has been constructed, tested and validated. The results obtained in the MCA on a uniform membrane composition are shown to be in agreement with earlier experimental data. Furthermore, the new electrostatic properties of the peptide, mimicking the PD of a peripheral membrane protein, have been revealed. These include the probability density functions of lipid association with the peptide (Fig. \ref{fig:occupation_probabilities}), the peptide total charge (Fig. \ref{fig:total_peptide_charge}) and the association times of lipids with the peptides (Fig. \ref{fig:lipid_association_times}). Finally, the reduction of the peptide diffusion coefficient upon the lipid sequestration has been explained by means of electrostatic properties of the peptide-lipid complex.
 \item Using the constructed MCA a novel hypothetical mechanism of the directional peptide drift on the heterogeneous membrane with the lipid gradient has been predicted. Namely, the basic peptide tends to drift to the area of the higher lipid density under the action of the force created by the gradient of the electrostatic potential. This mechanism can have a lot of biological implications (section \ref{future_directions} below). Additionally, the results show that the PIP$_2$ lipids are able to regulate the direction of the peptide drift.
 \item The novel CM has been constructed, based on the results of the MCA. The main advantage of the CM over the MCA is that it can be used to study the membrane lateral dynamics on the large (micrometer) scales. The CM has been tested and validated by comparison of its results with the results of the MCA. As a consequence of the peptide directional drift described in the MCA, the CM predicts a local accumulation of the peptide species due to the gradient of the negatively charged membrane lipids (Fig. \ref{fig:peptide_grad1}).
\end{enumerate}

The constructed computational models significantly contribute to the subject of the membrane modeling. A very small number of studies theoretically describe lateral dynamics of lipids and proteins on the cell membrane under the action of electrostatic forces. Recently, Khelashvili et al. \cite{Khelashvili2008} simulated the diffusion of a macroion on the membrane using Monte-Carlo simulation for protein movements and Cahn-Hilliard theory for the lipid dynamics. They have shown that electrostatic sequestration of negatively charged lipids in the vicinity of the moving microion strongly reduces its lateral diffusion. In fact their model predicts that only extremely mobile microions (with the diffusion coefficient 10 times larger than that of lipids) are able to perform lateral movements, whereas slow macroions practically never escape from the lipid shell in the vicinity of the peptide, formed due to electrostatic sequestration. However, as has been shown in subsection \ref{protein_lipid_dynamics_intoduction}, the measured protein diffusion coefficients are usually comparable to the lipid ones, meaning that model presented by Khelashvili et al. is not applicable to the description of real biological processes. The second major paper on protein membrane dynamics was presented by Hinderliter et al. \cite{Hinderliter2001}. In this study Monte-Carlo simulations of multiple proteins on a lattice are performed and, in addition to adsorption and desorption, protein diffusion is also introduced. Protein domain formation was obtained by varying the energy of interaction between monovalent negatively charged PS lipids and positively charged proteins. However, how the electrostatic interactions of proteins with underlying lipids contribute to the protein lateral dynamics was not considered in detail. 

The computational models presented in this thesis are deprived of all the limitations related to the models mentioned above. For example, the MCA provides the detailed description of the electrostatic nature of the membrane dynamics. Moreover, the problem of the restricted diffusion of the protein due to the associated lipid shell is resolved in the MCA by the lipid dragging mechanism. Additionaly, the MCA allows one to further develop it by introduction of other interactions between lipids and proteins. Thus, the improved MCA can be then used in various biological problems, for obtaining of the system detailed characteristics. Since the CM has been shown to be in agreement with the MCA and at the same time it is very computationaly efficient, one could further use it in the description of essential membrane processes.

\section{Directions for future work}

\label{future_directions}

The two constructed MCA and CM models can be further improved and developed to study biologically important phenomena, that include electrostatic interactions on the cell membrane. The possible directions for the future work can be defined as follows:

\begin{enumerate}
 \item The constructed MCA can be used to identify the role of dense protein clusters in the spatial dynamics of lipids. Multiple proteins with extended polybasic domains, such as MARKCS and GAP-43, or positively charged cytoskeletal polymers like septins are thought to form membrane micro and macro domains that sequester significant amounts of negatively charged lipids, in particular PIP$_2$. However, simple reaction-diffusion models demonstrate that within protein clusters, the concentration of free PIP$_2$ should be lower than in the surrounding protein-free membrane. The constructed MCA can be utilized to thoroughly investigate the dynamics of PIP$_2$ and monovalent lipids within such clusters at various densities and configurations of the positive charge distribution. The influence of such protein clusters on the diffusion of lipids can be also evaluated.
 \item The constructed CM model can be further extended to study various membrane mechanisms. For example, the protein membrane-cytoplasmic shuttling can be included to the system to account for the protein adsorption and desorption, allowing one to simulate the dynamics of the system at large time scales. Relevant protein phosphorylation/dephosphorylation by various enzymes can be also be included to simulate the changes of the intrinsic charge of the PDs. Additionally, the dynamics of the peptide-lipid complexes consisting of both PS and PIP$_2$ lipids (Fig. \ref{fig:peptide_complexes}) can be described in details.
 \item The modified and extended CM model, which allows to simulate the membrane diffusion at large lateral and temporal scales, can be used to analyze membrane domain formation events, that are potentially governed by the electrostatic interactions. For example, it is known that during clathrin-mediated endocytosis a large number of charged molecules, such as PIP$_2$ and PA, are produced and they regulate each other on the inner leaflet of the membrane. Thus, electrostatics can play a crucial role in this process and, therefore, the CM can be successfully utilized to study the endocytic cup initiation.
\end{enumerate}
